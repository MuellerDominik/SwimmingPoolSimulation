\section{Simulation}
\label{sec:Simulation}
Die Simulation verfügt über einen Tank, welcher den Pool modelliert, eine Umwälzpumpe und eine Heizung. Des Weiteren wird ein Thermostat zur Temperaturregelung und ein thermisches Netzwerk zur Modellierung der Umgebung eingesetzt.

Um die Simulation übersichtlicher zu Gestalten wurde diese in Teilsysteme unterteilt. Diese werden im nachfolgenden Unterkapitel genauer erläutert.

\subsection{Teilsysteme}
\label{subsec:Teilsysteme}
Die Simulation verfügt über die folgenden drei Teilsysteme:

\paragraph{Thermostat}
Der Thermostat wurde mit einer MATLAB-Funktion realisiert. Dazu wurde eine \texttt{persistent} Variable verwendet, welche ihren Wert zwischen Funktionsaufrufen beibehält und den Heizzustand speichert. Weiter wird geprüft, ob die aktuelle Temperatur innerhalb eines definierten Bereichs liegt. Dies entspricht einem einfachen Zweipunktregler mit einer Hysterese.

\paragraph{Heizung}
Die Heizung wurde mit einem elektrischen Durchlauferhitzer modelliert. Dazu wird während dem Heizvorgang die entsprechende Spannung an einem Leistungswiderstand angelegt. Die entstehende Wärme wird dann über ein 0.5\,m langes Kupferrohr mit einem Durchmesser von 10\,cm durch Wärmeleitung in das Wasser übertragen. Die Leistung des Durchlauferhitzers entspricht der berechneten Leistung der Wärmepumpe.

\paragraph{Umgebung}
Die Umwelt beeinflusst mit vier verschiedene Faktoren die Pooltemperatur:
\begin{itemize}
	\item Temperaturabnahme durch Verdunsten von Wasser.
	\item Temperaturzunahme oder Temperaturabnahme durch Konvektion über die Luft.
	\item Temperaturzunahme oder Temperaturabnahme durch Wärmeleitung an die Poolwände.
	\item Temperaturzunahme durch die Sonnenstrahlen.
\end{itemize}
Diese vier Faktoren befinden sich im Element Environment und sind parallel geschaltet. Das Vorgehen der vier Blöcke ist immer ähnlich. Es wird jeweils ein mathematisches Signal beschrieben, welches am Ende mit einem PS-Konverter in die physikalische Umgebung übersetzt wird. Danach wird das Signal mithilfe eines entsprechenden Blockes von Temperatur oder Leistung an das System angepasst. 

Das Verdunsten von Wasser hängt vor allem von der Sonne ab und macht einen grossen Anteil der Verluste aus \cite{WasserVerdunsten}. Deshalb erfolgt der Wärmeverlust entsprechend der Funktion der Sonnenstrahlung. Dabei wird dem Pool mit dem Matlab Element Controlled Heat Flow Source Energie entzogen. Die Energiemenge wurde mit den Arbeitsgrundlagen \ref{subsec:Annahmen} berechnet und der mittlere Wert der Funktion so gesetzt. Um einen Verlust zu erreichen, besitzt die Funktion ein negatives Vorzeichen.

Die Einstrahlung der Sonne wird gleich umgesetzt, wie die Verluste durch das Verdunsten. Die Funktion ist jedoch positiv , was zu einer Energiezufuhr führt. Ausserdem besitzt die Funktion eine andere, der Sonnenenergie entsprechende Amplitude. 

Um die Wärmeverluste oder Gewinne durch den Boden zu berechnen, bietet Simscape den Block Thermal Resistance. Damit kann auf beiden Seiten eine Temperaturdomaine angeschlossen, sowie der Widerstand dazwischen angegeben werden und Simscape simuliert den Energiefluss in J/s. Ausserdem wird eine Controlled Temperature Source verwendet, welche eine ideale Temperaturquelle darstellt. 

Mittels Convective Heat Transfer Block können die Konvektionsverluste an die Luft berechnet werden. Wie beim Boden wird auf beiden Seiten des Blockes eine Temperatur angegeben. Das Element benötigt zusätzlich noch die gesamte Pooloberfläche, sowie den Wärmeübergangskoeffizient. 